\documentclass{presentacion}

\usepackage[utf8]{inputenc}

\hypersetup{pdfpagemode=FullScreen}

\usepackage[spanish]{babel}
\usepackage{listings}
\usepackage[T1]{fontenc}

\usepackage{changepage} % provee adjustwidth

\usepackage{graphicx}
\usepackage{epstopdf}
\epstopdfsetup{update} % only regenerate pdf files when eps file is newer

\usefonttheme{structurebold} % using non standard fonts for beamer

\addtobeamertemplate{frametitle}{
   \let\insertframetitle\insertsectionhead}{}
\addtobeamertemplate{frametitle}{
   \let\insertframesubtitle\insertsubsectionhead}{}

\makeatletter
  \CheckCommand*\beamer@checkframetitle{\@ifnextchar\bgroup\beamer@inlineframetitle{}}
  \renewcommand*\beamer@checkframetitle{\global\let\beamer@frametitle\relax\@ifnextchar\bgroup\beamer@inlineframetitle{}}
\makeatother
   
\AtBeginSection[]
{  
  \begin{frame}
    \frametitle{Índice}
    \tableofcontents[currentsection]
  \end{frame}
}

\title[Contribuir en Open Source] 
{10 Formas Fáciles de Contribuir con Proyectos de Código Abierto}
\subtitle{Con Ejemplos de la Vida Real}
\author{Andrés Alcarraz}
\date{\today}

% configuración de tema
\usetheme{Madrid}
\usecolortheme{seahorse}

\usepackage{worldflags}
\flagsdefault[width=1em, framewidth=0px]

\usepackage{tikz}
\usebackgroundtemplate%
{
  \tikz[overlay,remember picture] 
  \node[opacity=0.2, at=(current page.south east),anchor=south east,inner sep=.3em] {
    \includegraphics[width=10em]{img/qr-code}
    \vspace{10em}
    };
}

\begin{document}
\frame{\titlepage}
\begin{frame}
 \begin{itemize} [<+->]
  \item ¿Quienes de ustedes usan proyectos open source?
  \item ¿Quienes quieren contribuir a proyectos open source? 
  \item ¿Por qué?
  \item ¿Quienes han contribuido?
 \end{itemize}

\end{frame}

\section{¿Por qué Contribuir?}
\begin{frame}
    Porque no solo de escribir código vive el desarrollador.
    \pause
    \begin{itemize}[<+->]
        \item \textbf{Aprendizaje:} Al contribuir vamos entendiendo el proyecto en profundidad.
        \item \textbf{Reconocimiento:} Hacerte conocido por mantenedores y usuarios
        \item \textbf{Comunidad:} Formar parte de algo más grande
        \item \textbf{Habilidades:} Construir experiencia para avanzar en tu carrera
        \item \textbf{Impacto:} Mejorar los proyectos para todos
    \end{itemize}
    \vspace{1em}
    \uncover<7->{\textbf{¡Solo necesitas estar dispuesto a hacerlo y prestar atención!}}
\end{frame}

\section{Empezando - Las Victorias Fáciles}

\subsection{Mejoras en Documentación}
\begin{frame}
    \begin{itemize}[<+->]
        \item Corregir errores tipográficos e instrucciones poco claras
        \item Agregar ejemplos faltantes
        \item Aclarar secciones confusas
        \item Documentar funcionalidades no documentadas
    \end{itemize}
    
    \vspace{1em}
    \uncover<5->
    {\textbf{Ejemplos:}}
    \begin {itemize}
        \item \small\link{https://github.com/JetBrains/kotlin-web-site/pull/4486}{Mejora de la documentación de kotlin}
        ( \small\link{https://kotlinlang.org/docs/kotlin-tour-control-flow.html\#loops}{Kotlin tour: control flow})
        \item \small\link{https://github.com/jpos/jPOS/pull/467}{Documentación de una funcionalidad extrayendo ejemplos del código fuente}
    \end {itemize}
\end{frame}

\subsection{Reporte de Errores}
\begin{frame}
    \begin{itemize}[<+->]
        \item Usar el software, encontrar problemas
        \item Proveer pasos claros para reproducir
        \item Incluir detalles relevantes del sistema
        \item Ser minucioso, necesitarás ser paciente para hacerlo bien
        \item Si tienes alguna idea, sugerir posibles soluciones. Si te sientes osado, puede ser en forma de PR.
    \end{itemize}  
    
    \vspace{1em}
    \uncover<6->{\textbf{Ejemplo:}
    \small\link{https://github.com/gradle/gradle-completion/issues/137}{Error de completado de Gradle en un proyecto multi-módulo recién creado}}
\end{frame}

\subsection{Traducciones}
\begin{frame}
    \begin{itemize}[<+->]
        \item Los hablantes nativos detectan errores de traducción fácilmente
        \item Hay problemas comunes: verbos que también pueden ser sustantivos dependiendo del contexto
        \item Iniciar la traducción si no está soportado
    \end{itemize}
    
    \vspace{1em}
    \uncover<4->{\textbf{Ejemplo:}
    \small\link{https://github.com/GSConnect/gnome-shell-extension-gsconnect/pull/1561}{Corrección en español de GSConnect}}
\end{frame}

\section{Participación en la Comunidad}

\subsection{Triaje de Issues}
\begin{frame}
    \begin{itemize}[<+->]
        \item Ayudar a los mantenedores a organizar issues
        \item Verificar problemas reportados
        \item Agregar información adicional
        \item Redirigir preguntas a los canales apropiados
    \end{itemize}
    
    \vspace{1em}
    \uncover<5->{\textbf{Ejemplo:} 
    \small\link{https://github.com/jpos/jPOS/issues/535\#issuecomment-1539188264}{Triaje de issue en jPOS}}
\end{frame}

\subsection{Soporte a Usuarios}
\begin{frame}
    \begin{itemize}[<+->]
        \item Unirse a canales de soporte (foros, listas de correo, chat)
        \item Responder preguntas basándose en tu experiencia
        \item Ayudar a solucionar problemas
        \item ¡No necesitas tener razón para ser útil!
    \end{itemize}
    
    \vspace{1em}
    \uncover<5->{\textbf{Ejemplo:} 
    \small\link{https://groups.google.com/g/jpos-users/c/ZHTk7mDKLy4}{Ayuda en la lista de correos de jPOS}}
\end{frame}

\section{Contribuciones Técnicas}

\subsection{Testing y Aseguramiento de Calidad}
\begin{frame}
    \begin{itemize}[<+->]
        \item Probar nuevas funciones en tu entorno
        \item Verificar que las correcciones de errores funcionen
        \item Ejecutar pruebas de regresión
        \item Suscribirse a notificaciones del repositorio
    \end{itemize}
    
    \vspace{1em}
    \uncover<5->{\textbf{Consejo:} Suscríbete a repositorios que te interesen para detectar problemas temprano}
\end{frame}

\subsection{Revisión de Código}
\begin{frame}
    \begin{itemize}[<+->]
        \item Revisar pull requests y commits
        \item Aprender leyendo el código de otros
        \item Detectar problemas potenciales
        \item Sugerir mejoras
    \end{itemize}
    
    \vspace{1em}
    \uncover<5->{\textbf{Ejemplo:} Después de dar 
    \small\link{https://github.com/jpos/jPOS/commit/2fb4fac06f76b16ebef95a699ba1a96271024f2c}{retroalimentación en un commit de jPOS}
    el autor \small\link{https://github.com/jpos/jPOS/commit/77b7bb68c887237eb9ef54e54bfc48ae941470be}{lo corrigió}}
\end{frame}

\subsection{Sugerencias de Funcionalidadess}
\begin{frame}
    \begin{itemize}[<+->]
        \item Proponer funcionalidades que necesitas
        \item Proveer casos de uso claros
        \item Sugerir enfoques de implementación
        \item Usar canales apropiados (discusiones, issues)
    \end{itemize}
    
    \vspace{1em}
    \uncover<5->{\textbf{Ejemplo:} Solicitudes de funciones para jPOS
    \small\link{https://github.com/jpos/jPOS/discussions/525}{para cargar ambientes desde diferentes directorios} ~
    \small\link{https://github.com/jpos/jPOS/discussions/546}{y para reemplazar globalmente propiedades de ambiente}}
\end{frame}

\section{Más Allá del Código}

\subsection{Contribuciones de Diseño y UX}
\begin{frame}
    \begin{itemize}[<+->]
        \item Sugerir mejoras visuales
        \item Crear mockups y diseños
        \item Proponer mejoras de flujo de trabajo
        \item UX no es solo sobre aspectos visuales
    \end{itemize}
    
    \vspace{1em}
    \uncover<5->{\textbf{Recuerda:} Incluso pequeñas sugerencias de UX pueden tener gran impacto}
\end{frame}

\subsection{Difundiendo la Palabra}
\begin{frame}
    \begin{itemize}[<+->]
        \item Escribir posts sobre proyectos que usas
        \item Compartir historias de éxito
        \item Presentar en conferencias y meetups
        \item Promocionar en redes sociales
    \end{itemize}
    
    \vspace{1em}
    \uncover<5->{\textbf{¡Esta presentación} está difundiendo la palabra sobre jPOS, GSConnect, y otros!}
\end{frame}

\section{Empezando Hoy}

\begin{frame}
    \begin{itemize}[<+->]
        \item Elige un proyecto que ya uses
        \item Comienza con documentación - es el punto de entrada más fácil
        \item Suscríbete a notificaciones
        \item Sé paciente y útil
        \item \textbf{¡Solo empieza!} Las pequeñas contribuciones importan
    \end{itemize}
    
    \vspace{2em}
    \uncover<6->{\large\textbf{Ahora ¿Cuántos de los que respondieron que no han contribuido se dieron cuenta de que en realidad sí lo habían hecho?}}
\end{frame}

\usebackgroundtemplate{}%
\section{Contacto}
\begin{frame}
    \begin{center}
        \vspace{1em}
        \email{alcarraz@gmail.com}\\
        \vspace{0.5em}
        \linkedin{andresalcarraz}\\
        \vspace{0.5em}
        \twitter{andresalcarraz}\\
        \vspace{0.5em}
        \github{alcarraz}\\
        \vspace{1em}
        
        \large\textbf{¡Gracias!}
    \end{center}
\end{frame}

\section{¿Preguntas?}
\begin{frame}
    \centering
    \includegraphics[height=0.8\textheight]{img/qr-code}
\end{frame}

\end{document}
